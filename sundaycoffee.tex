%% LyX 2.0.0 created this file.  For more info, see http://www.lyx.org/.
%% Do not edit unless you really know what you are doing.
\documentclass[english]{article}
\usepackage[latin9]{inputenc}
\usepackage{babel}
\begin{document}

\title{Sunday Coffee}


\author{Page Schmidt}

\maketitle
My final product, SundayCoffee, is not the program that I had in mind
when I began brainstorming projects for this course. However, in the
next few pages I'll describe why I still value the time I spent on
the project. I will also provide a description of the evolution of
my project over the course of this summer session and my goals for
building on it in the future.

Originally, my plan was to develop a comprehensive scheduler which
provided an intuitive interface for keeping track of one's everyday
responsibilities and projects. However, as I began to delve into the
details of putting together an application of that size, I realized
that my fantasies did not correllate very well with the realities
of a 6-week course, full-time job, and several inopportunely scheduled
weddings.

There were several major factors that contributed to the change of
scope of my project over the past few weeks, the main one simply being
my inexperience. Without a lot of other projects to draw comparisons
from, I found it difficult to guage what I could reasonably complete
in a given time frame. Tasks that I thought could be knocked out in
a couple of hours ended up taking me much longer due to technical
difficulties or other factors, and this ultimately set me back quite
a bit.

Another reason for the change in scope over the course of this summer
session was the steep learning curve that I had to overcome before
even beginning development on this project. The technologies I ultimately
ended up employing were PHP, MySQL, and AJAX, three tools that I hadn't
worked with previous to starting this course. Although I eventually
stumbled upon a very useful software bundle called xampp, I spent
a substantial amount of time trying to individually put together all
of the components required to make my PHP pages {}``work.''

Although my project didn't contribute directly to an active open-source
project, I was able to appreciate the benefits of freely available
code and collaborative learning. I relied on several sources of code
that had been created specifically for teaching the basics of putting
together a PHP application for the first time.
\end{document}
